\documentclass{article}
\usepackage{mathpkgs}
%\newcommand{\Im}{\text{Im}}
\newcommand{\Gr}{\text{Gr}}
\begin{document}
\section{Higgs Bundle Arising from Geometric Variation of Hodge Structure}
Now suppose that $(V^o, \nabla^o)$ is part of a polarized variation of Hodge structure.
Let $F^o$ be the Hodge filtration on $V^o$.
Suppose the monodromy of $\nabla^o)$ is unipotent, then we can extend the filtration $F^o$ 
to a filtration $F$ on $V_*$ by a theorem of Schmid:
\[
    F^pV_* = \pushforward{j}(F^{op}V^o)\cap V_*
\]
Let $E = \Gr_FV_*$, and $\theta = \Gr_F\nabla_*$, then $(E, \theta)$ is a Higgs bundle with
wht following properties
\begin{enumerate}
\item The Chern classes of $E$, in rational cohomology, all vanish.
\item The Higgs bundle is semistable in the sense that $\mu(E^\prime) \le \mu(E) = 0$
    for any proper coherent subsheaf stable under $\theta$
\item The Higgs field $\theta$ is nilpotent. This follows from the Griffith transversality
    of $\nabla_*$
\end{enumerate}

If the monodromy of $\nabla^o$ is only quasi-unipotent, then we need an intermediate step 
to construct the extension of $F^o$. Let $\pi: Y \rightarrow X$ be the cyclic cover in section 2.
The monodromy of $\pullback{\pi}\nabla^o$ is unipotent. Therefore, we can extend 
$\pullback{\pi}F^o$ to a filtration $\bar{F}$ on $\bar{V}$. Let
\[
    \phi: (\pushforward{\pi}\bar{V})^{G} \rightarrow V_*
\]
be the isomorphism {\color{red}section 2}. Then, we define
\[
    F^pV_* := \phi((\pushforward{\pi}\bar{F})^G)
\]
We will prove that
\begin{lemma}
$(E, \theta) = (\Gr_FV_*, \Gr_F\nabla_*)$ is a Higgs Bundle, \emph{i.e.} $\theta\wedge\theta = 0$.
\end{lemma}

\begin{lemma}
The parabolic Chern classes of $E$ is zero
\end{lemma}

\begin{lemma}
$E$ is parabolic semistable.
\end{lemma}

\begin{lemma}
$\theta$ is nilpotent, \emph{i.e.} $\nabla_*$ has Griffith transversality with respect to $F$.
\end{lemma}
\begin{proof}
Let $\alpha$ be a section of $V_*$, and let $\beta$ be the section of $(\pushforward{\pi}\bar{V})^G$,
such that $\phi(\beta) = \alpha$. We will show that $\bar{\nabla}(\beta)$ is $G$-invariant,
and $\phi(\bar{\nabla}(\beta)) = \nabla_*(\alpha)$.

\renewcommand{\vector}[2]{
    \left[\begin{array}{c}
            {#1}\\
            {\vdots}\\
            {#2}
        \end{array}\right]
}
It is enough to check locally. Use the notations from the proof of lemma 2.4. Locally on $U$, let
$\alpha$ be given by 
\[
    \vector{\alpha_1}{\alpha_r}
\]
with respect to the frame $<s_1, \cdots, s_r>$. Then, $\beta$ is given by
\[
    \vector{\alpha_1\prod\limits_{j=1}^sy_j^{d^j_1}}{\alpha_r\prod\limits_{j=1}^sy_j^{d^j_r}}
\]
with respect to the frame $<e_1,\cdots,e_r>$.
To make the notation simpler, we write $\prod\limits_{j=1}^sy_j^{d^j_i}$ as $y^{\delta_i}$ 
Then,
\[
    \bar{\nabla}(\beta) = \vector{y^{\delta_1}d\alpha_1 + \alpha_1 y^{\delta_1}\sum\limits_{j=1}^sd^j_1\frac{dy_j}{y_j}}
                          {y^{\delta_r}d\alpha_1 + \alpha_1 y^{\delta_r}\sum\limits_{j=1}^sd^j_r\frac{dy_j}{y_j}}
    + N_1\beta\frac{dy_1}{y_1} + \cdots N_s\beta\frac{dy_s}{y_s}
\]
Reorganize the terms, we have
\[
    \bar{\nabla}(\beta) = \vector{y^{\delta_1}d\alpha_1}{y^{\delta_r}d\alpha_r}
        + (N_1 + D_1)\beta\frac{dy_1}{y_1} + \cdots (N_s + D_s)\beta\frac{dy_s}{y_s}
\]
We observe that in $\bar{\nabla}(\beta)$ every coefficient in $\bar{V}$ is $G$-invariant. 
Therefore, $\bar{\nabla}(\beta)$ is a $G$-invariant section. 

Recall in the proof of lemma2.4, 
\[
    N_i + D_i = B_i = k_iNA_i
\]
So $\bar{\nabla}(\beta)$ can be written more compactly as
\[
    \bar{\nabla}(\beta) = \vector{y^{\delta_1}d\alpha_1}{y^{\delta_r}d\alpha_r} 
        + k_1NA_1\frac{dy_1}{y_1} + \cdots k_sNA_s\frac{dy_s}{y_s}
\]
As $y^{k_iN}_i = x_i$, we have
\[
    \frac{dy_i}{y_i} = \frac{dx_i}{k_iNx_i}
\]
Hence, 
\begin{align*}
    \phi(\bar{\nabla}(\beta)) & = d(\alpha) + A_1\alpha\frac{dx_1}{x_1} + \cdots  + A_s\alpha\frac{dx_s}{x_s} \\
                              & = \nabla_*(\alpha)
\end{align*}

As $\bar{\nabla}(\beta) \in \bar{F}^{p-1}\bar{V}$, we have proved that $\nabla_*{\alpha} \in F^{p-1}V_*$. 


\end{proof}
\end{document}
