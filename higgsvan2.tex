\documentclass{amsart}
\usepackage{verbatim,amssymb,amsmath,amscd,latexsym,amsbsy,mathrsfs} 
\usepackage{graphicx}
\usepackage[all]{xy}
 \input{xy}
 \xyoption{all}
\usepackage{color}
\usepackage{xparse}
\usepackage{mathtools}

\newtheorem{thm}{Theorem}[section]
\newtheorem{cor}[thm]{Corollary}
\newtheorem{df}[thm]{Definition}
\newtheorem{prop}[thm]{Proposition}
\newtheorem{corollary}{Corollary}
\newtheorem{lemma}[thm]{Lemma}
\newtheorem{rmk}[thm]{Remark}
\newtheorem{qsn}[thm]{Question}
\newtheorem{con}[thm]{Conjecture}
\newtheorem{ex}[thm]{Example}
\newtheorem{quest}[thm]{Question}
\newtheorem{prob}[thm]{Problem}


\DeclareMathOperator*{\im}{im} 
\DeclareMathOperator*{\coker}{coker} 
\DeclareMathOperator*{\rank}{rank} 
\DeclareMathOperator*{\Spec}{Spec}
\DeclareMathOperator*{\bSpec}{\mathbf{Spec}}
\DeclareMathOperator*{\Ind}{Ind }
\DeclareMathOperator*{\Res}{Res }
\DeclareMathOperator*{\image}{im}
\DeclareMathOperator*{\into}{\hookrightarrow}
\DeclareMathOperator*{\DR}{DR }
\DeclareMathOperator*{\Char}{char}
\DeclareMathOperator*{\Fr}{{{Fr}}}
\DeclareMathOperator*{\Kos}{Kos}
\DeclareMathOperator*{\parc}{par-c} 
\DeclareMathOperator*{\parch}{par-c} 

\newcommand{\integer}{\mathbb{Z}}
\newcommand{\chash}{\mathcal{\#}} 
\newcommand{\NN}{\mathbb{N}}
\newcommand{\RR}{\mathbb{R}} 
\newcommand{\QQ}{\mathbb{Q}}
\newcommand{\ZZ}{\mathbb{Z}} 
\newcommand{\CC}{\mathbb{C}} 
\newcommand{\Aff}{\mathbb{A}}
\newcommand{\PP}{\mathbb{P}} 
\newcommand{\FF}{\mathbb{F}}
\newcommand{\SSS}{\mathbb{S}} 
\newcommand{\TT}{\mathbb{T}}
\newcommand{\cB}{\mathcal{B}} 
\newcommand{\cF}{\mathcal{F}}
\newcommand{\cP}{\mathcal{P}} 
\newcommand{\cC}{\mathcal{C}}
\newcommand{\cA}{\mathcal{A}} 
\newcommand{\cS}{\mathcal{S}}
\newcommand{\cO}{\mathcal{O}} 
\newcommand{\cX}{\mathcal{X}}
\newcommand{\cp}{\mathpzc{p}}
\newcommand{\cV}{\mathcal{V}} 
\newcommand{\cD}{\mathcal{D}}
\newcommand{\cE}{\mathcal{E}} 
\newcommand{\cL}{\mathcal{L}} 
\newcommand{\cEnd}{\mathcal{E}nd}
\newcommand{\cK}{\mathcal{K}} 
\newcommand{\cm}{\mathpzc{m}}
\newcommand{\cM}{\mathcal{M}}
\newcommand{\cG}{\mathcal{G}} 
\newcommand{\minus}{\mathcal{n}}
\newcommand {\X} {\mathcal{X}}
\newcommand{\dirlim} {\displaystyle \lim_{\longrightarrow}}
\newcommand{\invlim} {\displaystyle \lim_{\longleftarrow}}
\newcommand{\D}{\mathcal{D}}
\newcommand {\V} {\mathcal{V}}
\newcommand {\bV} {\overline{\mathcal{V}}}
\newcommand {\bnabla} {\overline{\nabla}}
\newcommand {\bF} {\overline{F}}
\newcommand {\cY} {\mathcal{Y}}
\newcommand {\Om} {\tilde \Omega}
\newcommand {\cH} {\mathcal{H}}
\newcommand {\T} {\mathcal{T}}

\newcommand{\onto}{\twoheadrightarrow}

\newcommand {\N} {{\mathbb N}}
\newcommand {\C} {{\mathbb C}}
\newcommand {\R} {{\mathbb R}}
\newcommand {\Z} {{\mathbb Z}}
\newcommand {\Q} {{\mathbb Q}}
\newcommand {\HH} {{\mathcal H}}
\newcommand {\F} {{\mathcal F}}
\newcommand {\ga} {{\mathcal G \mathcal A}}
\newcommand {\B} {{\mathbb B}}
\newcommand {\E} {{\mathcal E}}
\newcommand {\dt} {{\bullet}}
\newcommand {\G} {{\mathbb G}}
\newcommand {\M} {{\mathcal M}}
\newcommand {\OO} {{\mathcal O}}
\newcommand {\A} {\mathbb{A}}
\newcommand {\I} {\mathcal{I}}
\newcommand{\pullback}[1]{{#1}^*}
\newcommand{\pushforward}[1]{{#1}_*}


\DeclarePairedDelimiter\floor{\lfloor}{\rfloor}

%
 \begin{document}
\title{Notes on parabolic Higgs bundles}
% \author{
%         ---   
% }
% \thanks {Partially supported by the NSF }
\address{Department of Mathematics\\
 Purdue University\\
 West Lafayette, IN 47907\\
U.S.A.}
 \maketitle

\section{parabolic bundles}

Let $X$ be a smooth projective complex variety with a reduced simple  normal crossing
divisor $D=\sum D_i$. Let $j:U=X-D\to X$ denote the inclusion of the complement. 
We fix this notation throughout the paper.
For our purposes, a parabolic bundle on $(X,D)$ consists of a
vector bundle $E$ on $X$ with an increasing $\R$-indexed filtration $E_\alpha\subset E(*D)$,
 by locally free $\OO_X$-modules such that
\begin{enumerate}
\item[P1.] $E_0=E$
\item[P2.] $E_{\alpha+1} = E_\alpha(D) $
\item[P3.] $E_{\alpha+c} = E_{\alpha}$ for some
  $c>0$ independent of $\alpha$.
\item[P4.] $Gr_\alpha E:= E_{\alpha}/E_{\alpha-\epsilon}$, $0<\epsilon\ll
  1$, is a locally free $\OO_D$-module. 
\end{enumerate}
This  definition is equivalent, with minor changes in notation, to the
definition by Yokogawa  \cite[3.1]{yokogawa}. 
%The last axiom is omitted in some
%references but it appears to be important;
% it implies condition \cite[3.2(1)]{biswas} needed below.)
These conditions ensure that the filtration has a finite number of
jumps in an interval, i.e.  values $\alpha$ such that $Gr_{\alpha_i}E\not=0$. 
We arrange the jumps in $[0,1)$ in increasing order  $0\le \alpha_1< \alpha_2<\ldots \alpha_\ell <1$.
These numbers are called weights.
%$$E_{\alpha_i}\subsetneqq E_{\alpha_{i+1}}$$ 
The  subsheaves $E_{\alpha_i}\subset E(D)$ determine the whole filtration. 
Setting $E_i= E_{\alpha_i}$ gives a finite filtration on $E(D)$ called a
quasiparabolic structure. From this point of view, a parabolic
bundle  is  a bundle with a quasiparabolic structure plus a choice of
weights. Parabolic bundles will be denoted by $E_*$.


We describe a few  basic examples. 

\begin{ex}\label{ex:trivial}
 Any vector bundle $E$ can be given a parabolic structure with integral weights and  $E_i= E(iD)$. We refer to this
 as a trivial parabolic bundle.
\end{ex}


\begin{ex}\label{ex:paraline}
 Choose a line bundle $L$ and
coefficients $\beta_i \in [0,1)$ and let
\begin{equation}
  \label{eq:L}
L_\alpha= L(\sum \lfloor \alpha+\beta_i\rfloor D_i)  
\end{equation}
\end{ex}

Any parabolic line bundle is of this form. Zariski locally, any
parabolic bundle is a sum of parabolic line bundles.

\begin{ex}\label{ex:paraDel}
 Suppose that $(V^o,\nabla^o)$ is  a  vector bundle with an integrable connection with regular singularities. By Deligne \cite{deligne}
there exists a unique extension
$$\nabla_\alpha:V_\alpha\to \Omega_{X}^1(\log D)\otimes V_\alpha$$
with residues having real part in $[ -\alpha,1-\alpha)$. This again forms a parabolic bundle, that we refer to as the Deligne parabolic bundle.
If the monodromy of $\nabla^o$  around components of $D$ is unipotent, then $V_*$ has trivial 
parabolic structure. This is because the eigenvalues of the monodromy of $\nabla^o$ around 
components of $D$ can is given by $\exp(2\pi i\alpha)$. 
So if the monodromy is unipotent, $\alpha$ must be integers. 
\end{ex}

A parabolic Higgs bundle on $(X,D)$ is a parabolic bundle $E_*$ together
with holomorphic map
$$\theta:E\to \Omega_X^1(\log D)\otimes E$$
such that
$$\theta\wedge \theta=0$$
and
$$\theta(E_\alpha)\subseteq E_\alpha$$

\section{Biswas's correspondence}
We will assume in this section  that the weights are rational with denominator dividing a fixed positive integer $N$.
 Recall that Kawamata \cite[thm17]{kawamata} has constructed a smooth projective variety $Y$, and a Galois covering
$\pi:Y\to X$, such that $\pi^*D_i = k_iN (
\tilde D_i)$ for some $k_i>0$, where $\tilde D_i=(\pi^*D_i)_{red}$.
Let $G$ denote the Galois group. A $G$-equivariant vector bundle on
$Y$, is a bundle $p:V\to Y$ (viewed geometrically rather than as a
sheaf) on which $G$ acts compatibly with $p$.  

We list some basic classes of examples.

\begin{ex}\label{ex:pullbackVB}
 $X$, then $\pi^*V'$  can made into a $G$-equivariant bundle, so that the projections $p$
$$
\xymatrix{
 \pi^*V'\ar[r]\ar[d]^{p} & V'\ar[d]^{p} \\ 
 Y\ar[r]^{\pi} & X
/}
$$
are compatible with  the $G$-action. 
\end{ex}

\begin{ex}\label{ex:equiLB}
 The line bundle $\OO_Y(\tilde D_i)$ has an  equivariant structure compatible with the one on $\pi^*\OO_X(D_i)$ under the isomorphism
$\OO_Y(\tilde D_i)^{\otimes k_iN}\cong \pi^* \OO_X(D_i)$.
\end{ex}


\begin{thm}[Biswas {\cite{biswas}}]\label{thm:biswas}
  There is an equivalence $E_*\to \tilde E$ between the category of 
  parabolic bundles on $X$ with weights in $\frac{1}{N}\Z$ and
  $G$-equivariant  bundles on $Y$. 
\end{thm}

We recall the construction in one direction. Given an equivariant bundle $\tilde E$ on
$Y$, we obtain a parabolic bundle
$$E_\alpha = \pi_*(\E\otimes \OO_Y(\lfloor \alpha \pi^*D\rfloor))^G$$
where $ \lfloor \alpha \pi^*D\rfloor=\sum_i \lfloor \alpha k_iN\rfloor
\tilde D_i$.  


Suppose that $(V^o,\nabla^o)$ is a vector bundle with connection satisfying the assumptions of 
example \ref{ex:paraDel}. In addition suppose that the eigenvalues of the   monodromy around $D$
are $N$th roots of unity. Then the weights of the  Deligne  parabolic bundle lie in  
$\frac{1}{N}\Z$.
Furthermore $(\tilde V^o,\Box^o)=(\pi^*V^o,\pi^*\nabla^o)$ has unipotent local monodromies.
Let $(V_*,\nabla_*)$  and $(\tilde V,\Box )$ denote Deligne's extensions of $V^o$ and $\tilde V^o$.

We prove the following
\begin{lemma}\label{lemma2.4} 
There is an isomorphism of vector bundle
\[
    \phi : \pullback{\pi}V_* \rightarrow \tilde V
\]
sending flat sections of $\pullback{\pi}\nabla_*$ to flat sections of $\Box$.
\end{lemma}
\begin{proof}
We will describe this morphism locally, and show it glues.
Take a point $y \in Y$, and let $x = \pi(y)$. Let $W$ and $U$ be polydisc neighborhoods
of $y$ and $x$, with coordinates $(y_1,y_2,\cdots,y_d)$ and $(x_1,x_2,\cdots,x_d)$, respectively.
Without loss of generality, we may assume that $\tilde D_i$ is locally defined by $y_i$ and $D_i$
is locally defined by $x_i$. As $\pullback{\pi}D_i = k_iN\tilde D_i$, we have
\[
    x_i = y_i^{k_iN}
\]
On $U-D$ let $<s_1,\cdots,s_r>$ be a free basis of $V^o$,the connection matrix of $\nabla_*$ be 
\[
    A_1\frac{dx_1}{x_1} + A_2\frac{dx_2}{x_2} + \cdots + A_s\frac{dx_s}{x_s}
\]
The generalized eigenvalues of $A_i$ are $\alpha_{i}$, the weights of $V_*$ on $D_i$. 
So on $W$, with respect to the frame $<e_1,\cdots,e_r> = \pullback{\pi}<s_1,\cdots,s_r>$ 
the connection matrix of $\pullback{\pi}\nabla_*$ is 
\[
    k_1NA_1\frac{dw_1}{w_1} + k_2NA_2\frac{dw_2}{w_2} + \cdots + k_sNA_s\frac{dw_s}{w_s}
\]
Write $B_i = k_iNA_i$. Let $J(B_i)$ be the Jordan canonical form of $B_i$, and let $D_i$ be 
the invertible part of $J(B_i)$. Write $N_i = B_i - D_i$. Then, $N_i$ is nilpotent, and the
conection matrix of $\tilde V$, with respect to the frame $<e_1,\cdots,e_r>$ is
\[
    N_1\frac{dy_1}{y_1} + N_2\frac{dy_2}{y_2} + \cdots + N_s\frac{dy_s}{y_s}
\]
Let 
\[
    F = \exp{-(N_1\log y_1 + N_2\log y_2 + \cdots + N_s\log y_s)}
\]
Then, the flat sections of $\tilde V$ has coordinates
\[
    F_1, F_2,\cdots, F_r
\]
where $F_i$ is the $i$-th column of $F$.

Similarly, let 
\begin{align*}
    G & = \exp{-(B_1\log y_1 + B_2\log y_2 + \cdots + B_s\log y_s)} \\
      & = \prod\limits_{i=1}^{s}y_i^{-D_i}F
\end{align*}
Then, the flat sections of $\pullback{\pi}\nabla_*$ has coordinates
\[
    G_1, G_2,\cdots, G_r
\]
where $G_i$ is the $i$-th column of $G$.

Let $d_i^j$ be the $j$-th eigenvalue of $D_i$. The map
\begin{align*}
    \pullback{\pi}V_* & \rightarrow \tilde V \\
                  e_j & \mapsto \prod\limits_{j=1}^{s}y_j^{d_i^j}e_j
\end{align*}
sends flat sections to flat sections. Hence, it glues to a morphism of vector bundles
\[
    \phi: \pullback{\pi}V_* \rightarrow \tilde V
\]
\end{proof}


\begin{lemma}
$\tilde V$ admits a $G$-equivariant action, 
and Biswas' construction applied to  $\tilde V$ yields $V_*$.
\end{lemma}

\begin{proof}
Let $\phi: \pullback{\pi}V_* \rightarrow \tilde V$ be the morphism from \ref{lemma2.4}. 
Set $d_i$ to be the largest eigenvalue of $D_i$. Then,
\[
    \phi\otimes O_Y(\sum\limits_{i=1}^{s}d_i\tilde{D_i}) :
        \pullback{\pi}V_*\otimes O_Y(\sum\limits_{i=1}^{s}d_i\tilde{D_i})
        \rightarrow \tilde V
\]
is an isomorphism. Therefore, we can define an equivariant $G$-action on $\tilde V$ via
$\phi$. Explicitly, write the Galois group of $\pi: Y\rightarrow X$ as
\[
    G = \bigoplus\limits_{i=1}^{s}\integer/k_iN
\]
Let $\mu_i$ be a generator of $\integer/k_iN$, \emph{i.e.} a primitive $k_iN$-th root of unity.
Then
\[
    \mu_i\cdot e_j = y_i^{d_i^j}e_j
\]
where $d_i^j$ is the $j$-th eigenvalue of $D_i$.

The $G$-invariant sections of $\tilde V$ are precisely the $G$-invariant sections of 
$\pullback{\pi}V_*$, and $G$-invariant section of $\pullback{\pi}V_*$ are $V_*$. 
Therefore, we have the identification
\[
    (\pushforward{\pi}\tilde V)^G = V_*
\]

\newcommand{\parabolic}{(\pushforward{\pi}\tilde V\otimes O_Y(\floor{\alpha\pullback{\pi}D}))^G}
To show $\parabolic$ recovers the parabolic structure of $V_*$, we show it locally by 
decomposing them into sum of line bundles. Hence, it is enough to assume $V_*$ is
a line bundle. 

Use the notation from \ref{lemma2.4}. $A_i = \alpha_i$ for some rational number $\alpha_i$.
The parabolic structure of $V_*$ is thus
\[
    V_{*\alpha} = O_X(\floor{\alpha - \alpha_1}D_1 + \cdots + \floor{\alpha - \alpha_s}D_s)
\]
which is the parabolic structure of $\parabolic$. 
\end{proof}



Biswas's correspondence  extends to Higgs bundles as well \cite[thm 5.5]{biswas2}.
Now suppose that $(V^o,\nabla^o)$ is part of a polarized variation of Hodge structure {\color{red} etc.}


\section{Parabolic Chern classes}

Given a parabolic line bundle with notation as in \ref{ex:paraline}
$${\parc}_1(L_*)= c_1(L)\pm \sum\beta_i[D_i]$$
 {\color{red}At this point, we need to check signs. The correct sign is the one which makes
 $\pi^* \parc_1(L_*)=  c_1(\mathcal{L})$ true, where $\mathcal{L}$ corresponds to $L_*$ under Biswas.}

 Given a parabolic bundle $E_*$, the top exterior power $\det E$ carries an induced parabolic structure.
 Set $\parc_1(E_*)= \parc_1(\det E_*)$.
Fix an ample line bundle $H$ on $X$.  Let $d=\dim X$.
The parabolic degree of a parabolic bundle $E_*$ is $c_1(\det E_*)\cdot H^{d-1}$.
We can define (semi)stability of parabolic and parabolic Higgs bundles using this \cite{my, yokogawa}.
Under Biswas's correspondence semistable parabolic bundles (Higgs bundles) with rational weights correspond
to semistable equivariant (Higgs) bundles.
 
 
Given a parabolic bundle $E_*$, let $p:Fl(E)\to X$ denote the full flag bundle of $E$. The pullback $p^*E$ 
carries a filtration $F^i\subset E$ by subbundles such that  associated graded $G^i = F^i/F^{i+1}$ are line bundles.
The parabolic structure on $E$ can be pulled back to  a parabolic structure on  $p^*E$ along $\pi^*D$, 
and $G^i$ carry induced parabolic structures.

\begin{lemma}
The classes $c_i$ defined below
$$1+c_1+ c_2+\ldots = \prod (1+\parc(G^i_*))$$
are pullbacks of classes  $\parc_i(E_*)\in H^{2i}(X,\R)$.
\end{lemma}

Since the map $H^*(X)\to H^*(Fl(E))$ is injective, the above property determines the above classes.
The following is stated in \cite[4.6]{biswas2}.

\begin{lemma}
 Suppose that $E_*$ has weights in $\frac{1}{N}\Z$. Let $p:Y\to X$ and $\mathcal{E}$ be as in theorem \ref{thm:biswas},
 then $p^*\parc_i(E_*)= c_i(\E)$.
\end{lemma}

\begin{proof}
 {\color{red} Fill in}
\end{proof}


\section{Vanishing}

\begin{prop}
 Let $(E_*,\theta)$ be a semistable Higgs bundle with zero parabolic Chern classes. There exists
 a parabolic bundle $(E_*',\theta')$ with  the same properties and rational weights with  $(E,\theta)=(E', \theta')$.
\end{prop}

\begin{proof}
{\color{red} Fill in} 
\end{proof}

Given a Higgs bundle $(E,\theta)$, we have complex
$$DR(E,\theta) = E\stackrel{\theta}{\to} \Omega_X(\log D)\otimes E\to \ldots$$

\begin{thm}
 Let $(E_*,\theta)$ be a  semistable parabolic Higgs bundle on $X$ with vanishing Chern classes and with $\theta$ nilpotent.
 Then
 $$H^i(DR(E,\theta)\otimes L)=0$$
 for $i>d$.
\end{thm}
 
 
 
\begin{proof}[Sketch]
 

{\color{red}
 Use above prop to reduce to case, where $E$ has rational weights.
By Biswas, it 
corresponds to $G$-equivariant Higgs bundle $(\tilde E,\tilde \theta)$ on $Y$.
We can apply the first main theorem of \cite{arapura} to conclude
$$H^i(Y, DR(\tilde E,\theta)\otimes \pi^*L)=0$$
Now take $G$-invariants.}
\end{proof}

%Further ideas:
%\begin{enumerate}
%\item  It would be good to work out explicitly the case of a variation of Hodge
%structure with quasiunipotent local monodromies in some detail.
%
%\item If $(E_*, \theta)$ is a parabolic Higgs bundles with real weights and
%the assumptions (V1), (V2), (V3), then I think vanishing should still
%hold. The idea is that by fixing the quasiparabolic structure and perturbing the weights slightly, we can find 
%parabolic bundle  $\tilde E_*$ with rational weights and $E_0=\tilde E_0$,
%such that the assumptions (V1), (V2), (V3) still
%hold. The idea for semistability is explained on the last page of
%\cite{biswas} in the non Higgs case. For (V1), recall that
%$\parc_i(E_*)$ is a linear comb. of
%rational cohomology classes with coefficients involving $\alpha_i$.  Now use the following
% fact, whose proof I omit:
%
%\begin{lemma}
%  If $A$ is a rational $m\times n$ matrix, then for any  solution
%  of $Av=0$ with $v\in \R^n$ and $\epsilon>0$ we can find a  rational solution $Av'=0$, $v'\in
%  \Q^n$, with $||v-v'||<\epsilon$.
%\end{lemma}
%This will  show that $\parc_i(E_*)=0\Rightarrow \parc_i(\tilde E_*)=0$
%\end{enumerate}
%

 \begin{thebibliography}{99}
 \bibitem[A]{arapura} D. Arapura, Kodaira-Saito vanishing via Higgs bundles in
   characteristic $p$, Crelle's J. (to appear)

 \bibitem[B1]{biswas} I. Biswas, Parabolic bundles as orbifold bundles,
   Duke (1997)

 \bibitem[B2]{biswas2} I. Biswas, Chern classes of parabolic bundles,
   J. Math Kyoto (1998)
   
   \bibitem[D]{deligne} P. Deligne,
   
   
 \bibitem[IS]{is} J. Iyer, C. Simpson, A relation between the
   parabolic Chern characters of the de Rham bundles, Math Ann (2007)
 \bibitem[K]{kawamata} Y. Kawamata, Characterization of abelian
   varieties, Composito (1981)
 \bibitem[MY]{my} Maruyama, Yokogawa, Moduli of parabolic bundles,
   Math Ann (1992)
 \bibitem[Y1]{yokogawa1} Yokogawa, Compactification of moduli of parabolic sheaves and
moduli of parabolic Higgs sheaves, J. Math. Kyoto (1993)

\bibitem[Y2]{yokogawa} Yokogawa, Infinitessimal deformations of
  parabolic Higgs sheaves, Int. J Math (1995)
 \end{thebibliography}
\end{document}
