\documentclass{article}
\usepackage{mathpkgs}
\begin{document}
\newcommand{\V}{\mathcal{V}}
\newcommand{\invariant}{(\pushforward{p}V_Y)^{\Gamma}}
\newcommand{\invariantt}{(\pushforward{p}V_Y\otimes O_Y(\floor{-m\times t}\tilde{D}))^{\Gamma}}
\newcommand{\tD}{\tilde{D}}
As Prof.Arapura pointed out during our last meeting, the cyclic group action on the equivariant
bundle on $Y$ should remember the corresponding local system downstair. 
I will make it clear soon in this note,
and I will define a group action on the equivariant bundle that does the job, and we will see
how the Biswas's recipe recovers the parabolic structure downstair.
We will see the line bundle case first, so that idea can be easily presented without messy
notation, then we will go to general vector bundle case

\textbf{Line bundle case} Let $X = \Delta$ with coordinate $x$. Let $D$ be the origin.
Consider the parabolic line
bundle $L = O_X\cdot x^{i/n}$. Let the connection $\nabla$ on $L$ to be the natural one, then
the flat sections are $\V = \complex<x^{-i/n}\cdot x^{i/n}>$. To simplify things, we write
$\V = \complex<x^{i/n}>$ so that we have enough symbols to preserve the monodromy infomation. 

Let $p: Y \rightarrow X$ be defined by $y^n = x$. Let $\tD = (\pullback{p}D)_{\text{red}}$.
The Galois group $\Gamma$ of $p$ is the cyclic
group generated by $\mu_n$. $\inverse{p}\V$ is the trivial local system, so its canonical 
extension $V_Y$ is the trivial bundle $O_Y$ with the trivial connection. We define an action of 
$\Gamma$ on $O_Y$ such that it encodes the information that $V_Y$ is the canonical extension
of $\inverse{p}\V$ but not $\inverse{p}(\text{something like the trivial local system})$. 
Let $y$ be the local coordinate
on $Y$. The action we want is
\[
    \mu_n \cdot y^j = \mu_n^{j/i}\times y^j
\]
Then, the $\Gamma$-invariant part of $O_Y$ is $O_X\cdot y^i$. Hence
\[
    (\pushforward{p}V_Y)^{\Gamma} = L
\]


\textbf{General vector bundle case} Now let 
$V_X = \bigoplus\limits_{i=1}^{r} O_X\cdot x^{a_i/n_i}$ be a vector bundle of $r$ with the natural
connection $\nabla$. It is the canonical extension of 
$\V = \complex<x^{a_1/n_1},\cdots, x^{a_r/n_r}>$. Let $m$ be the product of $n_i$ and let 
$\alpha_i \in \integer$ such that
\[
    \frac{\alpha_i}{m} = \frac{a_i}{n_i}
\]

Assume for simplicity that $\alpha_i$ are not redundant and 
\[
    \alpha_1 < \alpha_2 < \cdots < \alpha_r 
\]

Let $p: Y \rightarrow X$ be defined by $y^m = x$. The canonical extension $V_Y$ of
$\inverse{p}\V$ is the trivial bundle $\bigoplus\limits_{i=1}^{r} O_Y\cdot e_i$ with
the trivial connection. Let $(y^{j_1}\cdot e_1, y^{j_2}\cdot e_2,\cdots, y^{j_r}\cdot e_r)$
be a section of $V_Y$. The $\Gamma$-action on $V_Y$ we want should look like
\[
    \mu_m \cdot y^{j_k}e_k = (\mu_m^{mj_k/\alpha_k}\times y^{j_k})\cdot e_k
\]

Just like the line bundle case, we can see that 
\[
    (\pushforward{p}(V_Y))^{\Gamma} = V_X
\]

The parabolic structure on $\invariant$ is given by $\invariantt$. 

Now, we see how $\invariantt$ recovers the parabolic structure on $V_X$ which is defined via 
the generalized eigenspace of $\text{Res}\nabla$. In our case, $\text{Res}\nabla$ is the 
diagonal matrix
\[
    \text{diag}(\alpha_1/m, \alpha_2/m,\cdots,\alpha_r/m)
\]
Let $A(i)$ the the eigenspace of $\text{Res}\nabla$, and let
$F_i = \bigoplus\limits_{j=i}^{r} A(j)$. Define the subsheaf $\bar{F_i}$ of $V_X$ via the 
exact sequence
\[
    0 \rightarrow \bar{F}_i \rightarrow V_X \rightarrow V_X|_D/F_i \rightarrow 0
\]
We can see that 
\begin{align*}
    \bar{F}_1 & = V_X \\
    \bar{F}_2 & = O_X(-D)\cdot x^{\alpha_1/m}\bigoplus\limits_{i=2}^{r}O_X\cdot x^{\alpha_i/m} \\
    \bar{F}_3 & = O_X(-D)\cdot x^{\alpha_1/m}\oplus O_X(-D)\cdot x^{\alpha_2/m}
        \bigoplus\limits_{i=3}^{r}O_X\cdot x^{\alpha_i/m} \\
              & \vdots \\
    \bar{F}_k & = \bigoplus\limits_{i=1}^{k-1}O_X(-D)\cdot x^{\alpha_i/m}
        \bigoplus\limits_{i=k}^{r}O_X\cdot x^{\alpha_i/m} \\
\end{align*}
The parabolic structure on $V_X$ is given by
\[
    V_X = \bar{F}_1 \supset \bar{F}_2 \supset \cdots \supset \bar{F}_r 
        \supset \bar{F}_{r+1} = V_X(-D) 
\]
with weights $\alpha_1/m, \alpha_2/m,\cdots,\alpha_r/m$.

To investigate the parabolic structure on $\invariant$, we consider first the parabolic
structure on each component $(\pushforward{p}O_Y\cdot e_i)^{\Gamma}$, \emph{i.e.} given the group 
action $\Gamma$ on $\pushforward{p}O_Y\cdot e_i$, we need to figure out when jumps happen for
\[
    (\pushforward{p}O_Y\cdot e_i\otimes O_Y(\floor{-m\times t}\tD))^{\Gamma}
\]
Set $i = 1$. The $\Gamma$-action is given by
\[
    \mu_m\cdot y^j = \mu_m^{mj/\alpha_1}\times y^j
\]
As we let $j$ increase and $j \le \alpha_1$, the first class of invariant sections showing up are
\[
    f(x)y^{\alpha_1}
\]
where $f(x)$ comes from downstair. 
Those sections push-forward to $O_X\cdot x^{\alpha_i/m}$. But as soon as $j > \alpha_1$, 
we will have to wait till $j = m + \alpha_1$ to see next class of invariant sections
\[
    f(x)xy^{\alpha_1}
\]
which push-forward to $O_X(-D)\cdot x^{\alpha_1/m}$ 

Putting everything together, we see that for $t \le \frac{\alpha_1}{m}$, 
\[
    \invariantt = V_X
\]
for $\frac{\alpha_1}{m} < t \le \frac{\alpha_2}{m}$
\[
    \invariantt = \bar{F}_1
\]

Iterate this line of argument, we see that for 
$\frac{\alpha_k}{m} < t \le \frac{\alpha_{k+1}}{m}$
\[
    \invariantt = \bar{F}_k
\]
This proves that $\invariantt$ does recover the parabolic structure on $V_X$.

\textbf{Nontrivial Generalized Eigenspaces Case}   
Let $\V$ be a quasi-unipotent local system on $X - D$ with monodromy $T$, 
where $T$ is a rank $r$ Jordan matrix. Let the generalized eigenvalue of $T$ be a primitive $m$-th
root of unity $\mu_m$ 


\end{document}
