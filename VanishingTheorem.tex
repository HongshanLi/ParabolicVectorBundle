\documentclass{article}
\usepackage{mathpkgs}
\begin{document}
\begin{theorem}[Kawamata's cyclic cover construction]
Let $X$ be a smooth projective complex variety, $D\subset X$ a normal crossing divisor. 
Let $D = \sum\limits_{i=1}^{k}D_i$ be the decomposation of $D$ into irreducible components.
Let $m_i$ be positive integers for $i = 1,\cdots, k$. Then, there exists a finite surjective
morphism $p: \tilde{X}\rightarrow X$ satisfying the following conditions
\begin{enumerate}
\item $\tilde{X}$ is smooth.
\item $\tilde{D} := \pullback{p}(D)_{\text{red}}$ is a normal crossing divisor on $\tilde{X}$.
\item Let $\pullback{D_i} = \sum\limits_{j}m_{ij}\tilde{D}_j$ be the decomposation into irreducible
compoments. Then, $m_i | m_{ij}$
\end{theorem}

\begin{corollary}[Local description of the cyclic cover map]
Let $D$ be a smooth irreducible divisor of $X$. Let $p: \tilde{X} \rightarrow X$ be the cyclic 
covering map branched over $D$ of degree $m$. Write $\tilde{D} = (\pullback{p}D)_{\text{red}}$.
Let $x \in D$ and $y \in \tilde{D}$ such that $p(y) = x$. Then, locally near $y$ and $x$, we can
find analytic neighborhoods $U$ and $V$ with coordinates $(y_1,\cdots,y_d)$ and $(x_1,\cdots,x_d)$
such that $p(U)\subset V$, and with respect to the coordinate systems, $p$ can be described as
\begin{align*}
    x_1 & = y_1^m \\
    x_2 & = y_2 \\
        & \vdots \\
    x_d & = y_d 
\end{align*}
\newcommand{\V}{\mathcal{V}}

\textbf{Simple case of Biswas' Correspondence} 
Consider the simple case when the normal crossing divisor on $X$ has only one compoment.
Let $D$ be a smooth irreducible divisor of $X$. Let $\V$ be a quasi-unipotent local system on
$X - D$. Let $V$ be the canonical extension of $\V$. If the monodromy of $\V$ is not unipotent, then $V$
will have nontrivial parabolic structure. 
Let $T$ be the monodromy of $\V$ around $D$. Let $m_1 \in \mathbb{N}$ such that $T^{m_1}$
is unipotent. Let $p: \tilde{X} \rightarrow X$ be the cyclic 
covering map branched over $D$ of degree $m$, such that $m_1 | m$.  
Write $\tilde{D} = (\pullback{p}D)_{\text{red}}$.

