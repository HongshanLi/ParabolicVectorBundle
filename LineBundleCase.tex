\documentclass{article}
\usepackage{mathpkgs}
\begin{document}
\newcommand{\tD}{\tilde{D}}
\renewcommand{\L}{\mathcal{L}}
Let $L :=O_X\cdot s$ be line bundle with connection $\nabla$ such that
\[
    \nabla(s) = \frac{i}{n}\frac{dx}{x}s
\]
And let $\L$ be local system of flat sections of $L$
Let $p: Y \rightarrow X$ be defined by $y^n = x$. The equivariant bundle on $Y$ should
be the Deligne extension of the pullback of the local system corresponding to $L$. 
In this case, the equivariant bundle should be the vector bundle $V_Y = O_Y\cdot e$ with 
the trivial connection. Let $\Gamma$ be the Galois group of $p$. $\Gamma$ is generated
by the $n$-th root of unity $\mu$. The $\Gamma$-action on $V$ should encode the local 
system on $X$. Last time I defined the action to be
\[
    \mu\cdot e = \mu^{-i}e
\]
which was a consequence of having the knowledge that $s = x^{i/n}$ in mind. But this time 
I have a better reason to justify this action, and we will see the jump happens at $i/n$.

First, consider $(V_1 :=\pullback{p}L, \nabla_1 = \pullback{p}\nabla)$. 
This sheaf carries a natural $\Gamma$-action. Write $V_1 = O_Y\cdot e_1$. Then, $\nabla_1$
can be represented as
\[
    \nabla_1(e_1) = i\frac{dy}{y}e_1
\]
The flat sections of $V_1$ is in fact $y^{-i}e_1$, which lives in $V_1(i\tD)$. This means
we can define an $O_Y$-isomorphism
\begin{align*}
    \alpha: V_1(i\tD) & \rightarrow V_Y \\
            y^{-i}e_1 & \mapsto e
\end{align*}
sending flat sections to flat sections. There is a natural $\Gamma$-action on $V_1(i\tD)$.
So we can define a $\Gamma$-action on $V_Y$ via $\alpha$, \emph{i.e.}
\[
    \mu\cdot e = \mu^{-i}\times e
\]

Next, we will see how
\newcommand{\invariant}{(\pushforward{p}V_Y)^{\Gamma}}
\newcommand{\invariantt}{(\pushforward{p}V_Y\otimes O_Y(\floor{-nt}\tD))^{\Gamma}} 
\[
    \invariantt
\]
recovers the parabolic structure of $L$. First, we need a map
\[
    \beta: \invariant \rightarrow L
\]
As a $O_X$-module, $\pushforward{p}O_Y$ looks like
\[
    \sum\limits_{j=0}^{n-1}O_X\cdot y^j
\]
and it has an algebra structure given by
\[
    y^n = x
\]
We write
\[
    \pushforward{p}V_Y = \sum\limits_{j=0}^{n-1}O_X\cdot y^j\otimes e
\]
Then, we know how $\Gamma$ acts on each direct summand. Obviously, the $\Gamma$-invariant part
is 
\[
    O_X\cdot y^i\otimes e
\]
So the map $\beta$ is
\begin{align*}
    \beta: \invariant & \rightarrow L \\
         y^i\otimes e & \mapsto s
\end{align*}


Write $V_Y\otimes O_Y(-k\tD) = O_Y\cdot y^k\otimes e$. Then, we write
\[
    \pushforward{p}V_Y\otimes O_Y(-j\tD) = \sum\limits_{j=0}^{n-1}
        O_X\cdot y^j\otimes y^k\otimes e
\]
we can see the invariant part is
\[
    O_X\cdot y^{j}\otimes y^k\otimes e
\]
such that $j+k = i, i+n, i+2n,\cdots$. That why as soon as $k$ is bigger than $i$, we will
jump to 
\[
    L(-D)
\]








\end{document}
 
