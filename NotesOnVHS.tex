\documentclass{article}
\usepackage{mathpkgs}
\begin{document}
Variation of Hodge structure consists of the following data
\begin{enumerate}
\item A connected complex manifold $M$
\item A flat complex vector bundle $H_{\complex} \rightarrow M$ with 
    a flat real structure $H_{\real}$, and a flat bundle of lattice
    $H_{\integer} \subset H_{\real}$, {i.e.} transition functions
    of $H_{\complex}$ a locally constant and real-valued. The flat connection
    of $H_{\complex}$ has real valued connection matrix

\item An integer $k$
\item A flat, nondegenerate bilinear form $S$ on $H_{\complex}$ which is rational
    with respect to $H_{\integer}$, \emph{i.e} restricting to each fibre, $S$
    is a locally constant matrix of rational coefficient with repect to the chosen
    basis $H_{\integer}$
\item A decreasing filtration
\[
    H_{\complex} \supset F^1\supset \cdots \supset 0
\]
by holomorphic subbundle.
\end{enumerate}
The objects need to satisfy the following
\begin{enumerate}
\item Over each point $t$, $H_{\complex}, F, S$ restrict to a polarized Hodge 
    structure of weight $k$.
\item $\nabla(F^p) \subset F^{p-1}$
\end{enumerate}

\newcommand{\HodgeNumber}{h^{p,q}}
\section{Classifying space for Hodge Structure}
Let $H_{\integer}$ be some lattice over $\integer$, and let $H_{\complex}$ 
be its complexification. Fix an integer $k$, and a collection of nonzero
integers $h^{p,q}$ such that $p + q = k$. First objective is to give a manifold
structure to the parameter space of all Hodge structure of weight $k$ on $H_{\complex}$.
Let $D_p = dim F^p = \sum\limits_{i\ge p} \HodgeNumber$. Then, we consider the product 
of Grassmannian
\[
    Grass = G(H_{\complex},D_0) \times G(H_{\complex}, D_1)\times\cdot \times 
\]
A possible filtration on $H_{\complex}$ can be considered as a point in $Grass$.
All possible filtrations on $H_{\complex}$ can be identified with the incidence
variety $I$ on $G$(why incidence variety is smooth)

The general linear subgroup of $H_{\complex}$ operates on $I$ transitively. 
The subset $F \subset I$ satisfying
\[
    H_{\complex} = F^p\oplus \bar{F}^{k-p+1}
\]
is an Zariski open subset of $I$. So $F$ parametrizes the Hodge structure on $H_{\complex}$.

Let $S$ be a nondegenerate bilinear form on $H_{\complex}$, symmetric or skew, depending
on the parity of $k$, such that 
\begin{align*}
    S(H^{p,q}, H^{r,s}) & = & 0, & \text{unless} p = s, q = r \\
    i^{p-q}S(v, \bar{v})& > & 0, & \text{if} v \in H^{p,q}, v\neq 0
\end{align*}

Let $\check{D}$ be the points of $I$ satisfy condition 1.
\[
    G_{\complex} = \{g \in GL(H_{\complex}) | s(gu, gv) = S(u, v)\}
\]
operates on $\check{D}$ transitively. 


Let $D \subset I$ be the subset of filtration satisfy the above two conditions.
$D$ is a open subvariety. 
\[
    G_{\real} = \{g\in GL(H_{\real}) | S(gu, gv) = S(u, v)\}
\]
acts transitively on $D$. Therefore, $D$ is smooth. $D$ parametrize Hodge 
structure of weight $k$ on $H_{\complex}$ with the polarization $S$

Choose a reference point $o \in D$, Let $H$ be the corresponding filtration.
Let $B \subset G_{\complex}$ that fixes $H$, \emph{i.e.} $gF^{p}(H) = F^{p}(H)$
for all $g\in B$ and all $p$. Then, we have a set-theoretical identification
\[
    G_{\complex}/B \cong \check{D}
\]

One obtains analogous identification
\[
    G_{\real}/V \cong D, V = G_{\real}\cap B
\]

\subsection{Group invariant subtangent bundle of $\check{D}$ and $D$}
The Lie algebras of $G_{\complex}$ can be described as
\[
    g = \{X \in GL(H_{\complex})|S(Xu, v) + S(u, Xv) = 0 \}
\]

How does an orthogonal matrix $M$ to an real-valued symmetric 
matrix look like, regarded as a bilinear form.

$g$ contains 
\[
    g_0 = \{X \in g | XH_{\real} \subset H_{R}\}
\]
$g = g_0\oplus ig_0$. Via the containment $G_{\real} \subset G_{\complex}$, $g_0$ 
becomes the Lie algebra of $G_{\real}$ . 
\[
    g = \bigoplus_p g^{p,-p}
\]
such that if $X \in g^{p,-p}$, $XH^{r,s} \subset H^{r+p, s-p}$. 

The Lie algebra $b$ of $B$ consists of all those $X \in g$ that preserves the reference Hodge
filtration. i.e
\[
    b = \bigoplus_{p\ge 0} g^{p,-p}
\]
Let $v$ be the Lie algebra of $V = G_{\real}\cap B$, then
\[
    v = g_0 \cap b = g_0\cap b \cap \bar{b} = b\cap g^{0,0}
\]

The holomorphic tangent space of $\check{D} \cong G_{\complex}/B$ at 
the base point is naturally isomorphic to $g/b$. Under this iso, the
action of the isotropy group $B$ on the tangent space correspond to
the adjoint action of $B$ on $g/b$. Consequently, the holomorphic tangent
bundle $T \rightarrow \check{D}$ coincides with the vector bundle associated
to the holomorphic principal bundle
\[
    B \rightarrow G_{\complex} \rightarrow G_{\complex}/B \cong \check{D}
\]
\[
    b \oplus g^{-1,1}/b 
\]
Defines a $B-$invariant subspace of $g/b$. Therefore, it gives rise to a $B$
invariant subbundle $T_h(\check{D})$. Call it horizontal tangent bundle.

A holomorphic map $\Psi: M \rightarrow \check{D}$ is said to be horizontal if 
at each point $M$, the induced map on tangent space takes value in $T_h(\check{D})$.

Construct $T_h(D)$ by restriction. $G_{\complex}$-invariance of $T_h(\check{D})$ 
implies $G_{\real}$-invariance of $T_h(D)$. 

\subsection{Alternate description of the horizontal tagent bundle}
\begin{lemma}
At each point $c \in \check{D}$, correpsonding to the filtration $F(c)$, a vector
field $X \in g$ takes value in the horizontal tangent space if and only if
$X$ regarded as an endomorphism of $H_{\complex}$, maps $F^p$ to $F^{p-1}$
\end{lemma}
\begin{proof}
At the base point, the value of $X$ lie in
\[
    b\oplus g^{-1,1}/b
\]
Therefore, the value of $X$ can be represented by somebody in $g^{-1,1}$
\end{proof}

By construction, $\check{D}$ carries a tautological complex vector bundle $H_{\complex}(\check{D})$,
with fibre $H_{\complex}$. The global filtration $\mathbb{F}$ of the bundle, restrict to the 
fitration of $H_{\complex}$ corresponding to the base point. Let
\[
    \nabla: \mathbb{H_{\complex}} \rightarrow \mathbb{H_{\complex}}\otimes \Omega_{\check{D}}
\]
be the flat connection. 
\newcommand{\hd}{H_{\complex{\check{D}}}}
Let $p:\hd \rightarrow \hd/F^p$, the composition
\[
    p\circ\nabla F^p \rightarrow \hd/F^p \otimes \Omega_{\check{D}}
\]
is $O_{\check{D}}$-linear. 

Consider the holomorphic map $\Psi: M \rightarrow \check{D}$ of a complex manifold $M$ into 
$\check{D}$. The vector bundle $H_{\complex}(\check{D})$ pulls back to the trivial bundle
$\mathbb{H_{\complex}} \rightarrow M$

A holomorphic map $\Psi: M \rightarrow D$ is horizontal if and only if 
\[
    \nabla(F^p) \subset F^{p-1}\otimes\Omega_{D}
\]

\newcommand{\VHS}{\mathbb{H}_{\complex}}
Now, consider a variation of Hodge structure $\{M, \VHS, F\}$. Let $\pi: \tilde{M} \rightarrow M$
denote the universal covering of $M$. The vector bundle $\VHS$ pullback to a trivial bundle
$\tilde{M}\times H_{\complex}$. The filtration also pulls back. Therefore, for each point 
$m \in \tilde{M}$, we get a Hodge structure on $H_{\complex}$ of weight rank $\VHS$. So we get
a map
\[
    \tilde{\Phi}: \tilde{M} \rightarrow D
\]
Variation of Hodge structure has the attribute of transversality. Therefore, the map $\tilde{\Phi}$
is automatically horizontal.

$\VHS.H_{\integer}$ and $\VHS.S$ are flat, therefore the action of $\pi_1(M)$ preserves both. 

One can think of $\VHS$ on $M$ as a quotient from the universal cover in the first place.

The subgroup
\[
    \Gamma = \phi(\pi_1(M)) \subset G_{\integer}
\]
is called the monodromy group of the variation of Hodge structure. 

By construction of $\tilde{\Phi}$, if two points of $\tilde{M}$ are 
related by some $\sigma \in \pi_1(M)$, the corresponding Hodge structures are related by
$\psi(\sigma)$.
\[
    \tilde{\sigma a} = \psi(\sigma)\circ\tilde{\Psi}(a), a \in \tilde{M}, \sigma \in \pi_1(M)
\]
Therefore, $\tilde{\Phi}$ descends to a mapping
\[
    \Phi: M \rightarrow D/\Gamma
\]
This map is called Griffith's period mapping for variaiton of Hodge structure. 

\begin{theorem}(Griffith)
The period mapping is holomorphic, locally liftable to $D$, and local liftings are horizontal
\end{theorem}

\section{Nilpotent Orbit Theorem}
Let $\phi: \Delta^* \rightarrow D/\Gamma$ be the period mapping of VHS on $\Delta^*$
Let $\pi: U \rightarrow \Delta^*$ be the universal covering map. $\phi\circ\pi$ is locally
liftable, and since $U$ is simply connected, it is globally liftable. We have the following diagram
\[
\begin{tikzcd}
    & U \arrow{r}{\tilde{\Phi}}\arrow{d}{\pi} 
        & D \arrow{d} \\
    & \Delta^* \arrow{r}{\Phi} 
        & D/\Gamma \\
\end{tikzcd}
\]
One can choose an element $\gamma \in \Gamma\subset G_{\integer}$, such that
\[
    \tilde{\Phi}(z+1) = \gamma\circ\tilde{\Phi}(z)
\]
\begin{lemma}(Borel)
All eigenvalues of $\gamma$ are roots of unity
\end{lemma}
\begin{proof}
Computed with respect to Poincar\'e metric 
\[
    \frac{dx^2 + dy^2}{y^2}
\]
on $U$. the points $in$ and $in + 1$ have the distance $1/n$.
According to 3.17, if a $G_{\real}$-invariant Riemannian distance function $d$
on $D$ is suitably renormalized, $\tilde{\Phi}$ will not increase distances.
When the identification $D \cong G_{\real}/\Gamma$ is made as in 3.6, each
pf the image points $\tilde{\Phi}(in)$ has a coset rep $g_nV$. So
$\tilde{\Phi}(in + 1) = \gamma\tilde{\Phi}(in)$ corresponds to the coset $\gamma g_nV$.
Thus
\begin{align*}
    d(\inverse{g_n}\gamma\g_n V, eV) & = & d(\gamma g_nV, g_nV) \\
                                     & = & d(\tilde{\Phi}(in+1), \tilde{\Phi}(in))
                                     & \le & \frac{1}{n}
\end{align*}
This means the congugacy class of $\gamma$ converge to a point in the compact subgroup
$V \subset G_{\real}$. To show eigenvalues of $\gamma$ have abosolute value 1, 
it is enough to show that the eigenvalues for each each element in $V$ have absolute 
value 1. As $\gamma$ is integer-valued matrix with respect to the basis $H_{\integer}$
it follows that the eigenvalues of $\gamma$ must be root of unity. 
\end{proof}

\newcommand{\Im}{\text{Im}}
\begin{theorem}(Nilpotent Orbit Theorem)
The mapping $\Psi$ can be continued holomorphiccaly over the puncture of $\Delta^*$.
The point $a = \Psi(0) \in \check{D}$ is a fixed point of $\gamma_s$. 
For a suitable constant $\alpha \ge 0$, $\Im z > \alpha$ implies that $\exp(zN)\circ\alpha
\in D$/ Perhaps after increasing the constant $\alpha$, and for a suitable choice of 
$\beta \ge 0$, $\Im z > \alpha$ also implies the inequality
\[
    d(\exp(zN))\circ a, \tilde{\phi}(z)) \le (\Im z)^{\beta}\exp(-2\pi\inverse{m}\Im z)
\]
here $d$ is the $G_{\real}$ invariant distance function on $D$. The mapping
$z \mapsto \exp(zN)\circ a$ of $\complex$ into $\check{D}$ is horizontal.
\end{theorem}
\begin{proof}
Let $H_{\complex} = \bigoplus H_0^{p,q}$ be the Hodge structure corresponding to the 
base point $o \in D$. We get an induced Hodge structure
\[
    g = \bigoplus g^{p,-1}
\]
Let $\theta: g \rightarrow g$ be the Weil operator
\[
    \theta X = (-1)^pX
\]
The 1 and -1 eigenspaces of $\theta$ will be denoted by, $t$ and $p$. Set
\[
    t_0 = t\cap g_0, p_0 = p\cap g_0
\]
The properties of $\theta$ imply that
\[
    g = t\oplus p, g_0 = t_0\oplus p_0
\]
\[
    [t,t]\subset t, [t,p]\subset p, [p,p]\subset t
\]
The fixed point of $\bar{\theta}$ is 
\[
    m_0 = t_0\oplus ip_0
\]

$m_0$ is a real form of $g$, \emph{i.e.}
\[
    g = m_0\oplus im_0
\]
Let $C$ be the Weil operator on $H_{\complex}$ corresponding to the reference Hodge
structure. Then,
\[
    (u, v) = S(Cu,\bar{v})
\]
defines a Hermitian inner product . I think
\[
    (u, v) = S(Cu, v)
\]
defines a Hermitian inner product. 

$C$ is an element of the group $G_{\real}$, whose adjoint action on $g$ coincide with
$\theta$.
\begin{definition}(Adjoint representation)
Let $G$ be a Lie group. $x \in G$, and let $g_x$ be its Lie algebra. For $g\in G$
\[
    x \maps gx\inverse{g}
\]
is the adjoint representation of $G$.
\end{definition}
$m_0$ is the intersection of $g$ with skew Hermitian transformation. 
$M \subset G_{\complex}$, which corresponds to the subalgebra $m_0 \subset g$ coincides 
with the connected component of the identity in the intersection of $G_{\complex}$ with
the unitary group. So $M$ is compact. Also, a compact real form in a connected, 
complex simisimple Lie group is always connected and is its own normalizer. Therefore, 
$M$ equals the full intersection of $G_{\complex}$ with the unitary group.

The intersection
\[
    K = M \cap G_{\real}
\]
is a compact subgroup of $G_{\real}$, whose Lie algebra $t_0 = m_0\cap g_0$ consists 
of skew Hermitian transformations, whereas the complement $p_0$ of $t_0$ in $g_0$ 
consists of Hermitian transformations. So The connected component of the identity 
in $K$ forms a maximal compact subgroup of the connect component  of the identity in
$G_{\real}$. 

$K$ is a maximal compact subgroup of $G_{\real}$ and it meets every connected 
component of $G_{\real}$.

\begin{lemma}
$V = M \cap B$, and $V \subset K$
\end{lemma}
\begin{proof}
$x \in V = G_{\real}\cap B$ leaves the subspace $H_0^{p,q}$ invariant. $x \in M \cap B$
also leave the subspaces invariant. And $x$ preserve inner product. With respec to the 
inner product, the reference Hodge decomp is orthogonal, so elements of $M\cap B$ leaves
the subspace $H_0^{p,q}$. Also, every $g\in M$ is self-adjoint
\[
    Cg\inverse{C}\bar{v} = \bar{gv}
\]
If $g\in M$ also happens to preserve the subspaces $H_0^{p,q}$, it commutes with $C$.
It must then respect the real lattice of $H_{\complex}$, hence belongs to $G_{\real}$.
\[
    M\cap B \subset G_{\real}
\]
\end{proof}

The $M$-orbit of the identity coset in $G_{\complex}/B \cong \check{D}$, becomes 
naturally isomorphic to $M/V$. Since $G_{\real}$ and $M$ have the same dimension
(both are real forms of $G_{\complex}$). The dimension of $M/V$ and $D = G_{\real}/V$
also agree, so $M/V$ must be an open orbit. Since $M$ is closed, $M/V$ must also be a 
closed orbit. Hence, $M$ operates transitively on $\check{D}$, and this gives the 
identification
\[
    D \cong M/V
\]

Regard the elements of $g$ as linear transformations on the vector space $H_{\complex}$.
The bilinear form
\[
    B(X, Y) = \text{trace}XY, X,Y\in g
\]
is clearly symmetric, invariant under the adjoint action of $G_{\complex}$ on $g$,
and defined over $\real$, relative to the real structure $g_0 \subset g$

\begin{lemma}
The bilinear form $-B$ polarizes the Hodge structure $g = \bigoplus g^{p.-p}$
\end{lemma}
\begin{proof}
Let $X \in g^{p,-p}$ and $Y \in g^{q,-q}$. The linear map $XY$ shift the component
$H_0^{p,q}$ nontrially, so $XY$ does not have nontiral eigenvalue, i.e. it has no trace.
With respect to the inner product, the element of $m_0$ are skew Hermitian. This makes
the eigenvalues of any $X \in m_0$ all purely imaginary, and if $X \neq 0$, not all
eigenvalues can vanish. Thus, $B$ is negative definite on $m_0$. One has
\[
    -B(\theta X, \bar{X}) > 0
\]
\end{proof}

So 
\[
    (X, Y) = -B(\theta X, \bar{Y}), X, Y \in g
\]
turns $g$ into a complex Hilbert space and $g_0$ into a real Hilbert space. 

The adjoint actions of $V$, $K$, and $M$ leave the inner product on $g$ invariant.

\begin{lemma}
If $T \in g_0$ is nilpotent, and if $T$ is expressed as $Y + Z$, with $Y \in t_0$ and
$Z \in p_0$, then 
\[
    ||T|| = \sqrt{2}||Y|| = \sqrt{2}||Z||
\]
\end{lemma}

The holomorphic tangent space to $\check{D} \cong G_{\complex}/B$ at the identity coset
is naturally isomorphic to the quotient of $g/b$; the quotient, in turn, is isomorphic
as a $V$-module to the orthogonal complement of $b$ in $g$, namely
\[
    c = \bigoplus\limits_{p>0}g^{p,-q}
\]
From $g$, $c$ inherits a $V$-invariant inner product. When this inner product is 
translated via $G_{\real}$ and $M$, one obtains, a $G_{\real}$-invariant Hermitian
structure $h( , )$on $D = G_{\real}/V$ and an $M$-invariant Hermitian structure $h_M$
on $\check{D}\cong M/V$. The corresponding distance function shall be denoted by $d$ 
and $d_M$.

For $g \in G_{\complex}$, let $l(g):\check{D} \rightarrow \check{D}$ denote the 
left translation by $g$, and $l(g)_*$ the differential of this map. Any 
$g\in G_{\complex}$ can be expressed as $g = mb$, with $m\in M$ and $b\in B$, because
$M$ operates transitively on $G_{\complex}/B$. Then,
\[
    l(g)_* = l(m)_*\circ l(b)_*
\]

\begin{lemma}
At each point of $\check{D}$, the operator norm of the linear transformation
$l(g)_*$, measured relative to the Hermitian structure $h_M$, is bounded
by the operator norm of $Adg$ acting on $g$.
\end{lemma}
 
The above statement implies a relation between the two metric $h$ and $h_M$ on $D$.
Let $X$ be a holomorphic tangent vector at the point $gV \in G_{\real}/V$ . 
Then there exists a holomorphic tangent vector $Z$ at the identity coset, such that
$l(g)_*Z = X$. By construction, the two metric $h$ and $h_M$ agree at the identity 
coset. With respect to $h$, $l(g)_*$ are isometries, with $h_M$, it is not. But the 
above statement limits the dialation. Hence:

\begin{corollary}
If the point $a \in D$ is the $g$-translate of the base point, with $g\in G_{\real}$,
and if $X$ is a holomorphic tangent vector at $a$, then
\begin{align*}
    h(X, X)^{1/2} & \le & ||Ad\inverse{g}|| h_M(X, X)^{1/2} \\
    h_M(X, X)^{1/2} & \le & ||Ad g}|| h(X,X)^{1/2}
\end{align*}
\end{corollary}

For each $z \in U$, I choose an element $g(z) \in G_{\real}$ whose $V$-coset represents
the point $\tilde{\phi}(z)\in D$. Although $g(z)$ is determined only upto right
multiplication by an element of $V$, which operates unitarily on $g$, the following 
statement is meaningful:
\begin{lemma}
There exist positive constant $\alpha$, $\beta$, which depends only on the choice
of the base point in $D$ and on the integer $m$, such that $\Im z > \alpha$ implies
\[
    ||Ad\inverse{g(z)}N|| \le \beta\inverse{\Im z}
\]
\end{lemma}
\begin{proof}
The tangent space to $G_{\real}/K$ at the identity coset is isomorphic as $K$-module
to $g_0/t_0$. This quotient inherits an inner product from $g_0$, which is invariant
under $K$ and it can thus be translated into a $G_{\real}$-invariant Riemmanian structure;
the Riemannian structure then defines a $G_{\real}$-invariant distance function 
\[
    d_{G_{\real}/K}
\]
because of the $G_{\real}$-invariance of the metrics of $D$, the quotient map
\[
    D \cong G_{\real}/V \rightarrow G_{\real}/K
\]
has uniformly bounded differential; by renormalizing the metric on $G_{\real}/K$, the
bound can be arranged to have value one. Now let $G_{\real}=UAK$ be an Iwasawa decomposation.
A is a vector subgroup of $G_{\real}$, and $U$ for a suitable maximal unipotent subgroup.
For each $z\in U$, I can choose $k(z)\in K$ such that
\[
    Ad\inverse{g(z)}N \in Ad\inverse{k(z)}u_0
\]
Because of the uniform boundedness property of $\tilde{\Phi}$, for a suitable positive
constant $C$, one finds
\end{proof}

\begin{lemma}
There exists a neighborhood $U$ of the base point in $\check{D}$, and positive constant
$\eta, C$ such that $X\in g$, $\norm{X} < \eta, a \in U$ together imply that
$\exp{X}\circ a \in D$, and $d(\exp{X}\circ a, a) < C\norm{X}$
\end{lemma}

I now define a mapping $F_z: U - z \rightarrow \check{D}$ by
\begin{align*}
    F_z(u) & = & \inverse{g(z)}\exp(-muN)\circ\tilde{\Phi}(z + mu) \\
\end{align*}
it is holomorphic and periodic of period one. Let $P$ be a polydisk neighborhood of 
the base point in $D$. After shrinking $P$, I may assume that the $G_{\real}$-invariant
metric of $D$ and the Euclidean metric of $P$ are mutually uniformly bounded. 

\begin{lemma}
There exists positive constants $\alpha, \zeta$, which depend only on the choice of base
point, the choice of $P$, and on the integer $m$, such that $F_z(u)\in P$ whenever
$\Im z > \alpha$ and $\abs{\Im u} < \zeta\Im z$
\end{lemma}

 






\end{document}
