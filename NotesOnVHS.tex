\documentclass{article}
\usepackage{mathpkgs}
\begin{document}
Variation of Hodge structure consists of the following data
\begin{enumerate}
\item A connected complex manifold $M$
\item A flat complex vector bundle $H_{\complex} \rightarrow M$ with 
    a flat real structure $H_{\real}$, and a flat bundle of lattice
    $H_{\integer} \subset H_{\real}$, {i.e.} transition functions
    of $H_{\complex}$ a locally constant and real-valued. The flat connection
    of $H_{\complex}$ has real valued connection matrix

\item An integer $k$
\item A flat, nondegenerate bilinear form $S$ on $H_{\complex}$ which is rational
    with respect to $H_{\integer}$, \emph{i.e} restricting to each fibre, $S$
    is a locally constant matrix of rational coefficient with repect to the chosen
    basis $H_{\integer}$
\item A decreasing filtration
\[
    H_{\complex} \supset F^1\supset \cdots \supset 0
\]
by holomorphic subbundle.
\end{enumerate}
The objects need to satisfy the following
\begin{enumerate}
\item Over each point $t$, $H_{\complex}, F, S$ restrict to a polarized Hodge 
    structure of weight $k$.
\item $\nabla(F^p) \subset F^{p-1}$
\end{enumerate}

\newcommand{\HodgeNumber}{h^{p,q}}
\section{Classifying space for Hodge Structure}
Let $H_{\integer}$ be some lattice over $\integer$, and let $H_{\complex}$ 
be its complexification. Fix an integer $k$, and a collection of nonzero
integers $h^{p,q}$ such that $p + q = k$. First objective is to give a manifold
structure to the parameter space of all Hodge structure of weight $k$ on $H_{\complex}$.
Let $D_p = dim F^p = \sum\limits_{i\ge p} \HodgeNumber$. Then, we consider the product 
of Grassmannian
\[
    Grass = G(H_{\complex},D_0) \times G(H_{\complex}, D_1)\times\cdot \times 
\]
A possible filtration on $H_{\complex}$ can be considered as a point in $Grass$.
All possible filtrations on $H_{\complex}$ can be identified with the incidence
variety $I$ on $G$(why incidence variety is smooth)

The general linear subgroup of $H_{\complex}$ operates on $I$ transitively. 
The subset $F \subset I$ satisfying
\[
    H_{\complex} = F^p\oplus \bar{F}^{k-p+1}
\]
is an Zariski open subset of $I$. So $F$ parametrizes the Hodge structure on $H_{\complex}$.

Let $S$ be a nondegenerate bilinear form on $H_{\complex}$, symmetric or skew, depending
on the parity of $k$, such that 
\begin{align*}
    S(H^{p,q}, H^{r,s}) & = & 0, & \text{unless} p = s, q = r \\
    i^{p-q}S(v, \bar{v})& > & 0, & \text{if} v \in H^{p,q}, v\neq 0
\end{align*}

Let $\check{D}$ be the points of $I$ satisfy condition 1.
\[
    G_{\complex} = \{g \in GL(H_{\complex}) | s(gu, gv) = S(u, v)\}
\]
operates on $\check{D}$ transitively. 


Let $D \subset I$ be the subset of filtration satisfy the above two conditions.
$D$ is a open subvariety. 
\[
    G_{\real} = \{g\in GL(H_{\real}) | S(gu, gv) = S(u, v)\}
\]
acts transitively on $D$. Therefore, $D$ is smooth. $D$ parametrize Hodge 
structure of weight $k$ on $H_{\complex}$ with the polarization $S$

Choose a reference point $o \in D$, Let $H$ be the corresponding filtration.
Let $B \subset G_{\complex}$ that fixes $H$, \emph{i.e.} $gF^{p}(H) = F^{p}(H)$
for all $g\in B$ and all $p$. Then, we have a set-theoretical identification
\[
    G_{\complex}/B \cong \check{D}
\]

One obtains analogous identification
\[
    G_{\real}/V \cong D, V = G_{\real}\cap B
\]

\subsection{Group invariant subtangent bundle of $\check{D}$ and $D$}
The Lie algebras of $G_{\complex}$ can be described as
\[
    g = \{X \in GL(H_{\complex})|S(Xu, v) + S(u, Xv) = 0 \}
\]

How does an orthogonal matrix $M$ to an real-valued symmetric 
matrix look like, regarded as a bilinear form.

$g$ contains 
\[
    g_0 = \{X \in g | XH_{\real} \subset H_{R}\}
\]
$g = g_0\oplus ig_0$. Via the containment $G_{\real} \subset G_{\complex}$, $g_0$ 
becomes the Lie algebra of $G_{\real}$ . 
\[
    g = \bigoplus_p g^{p,-p}
\]
such that if $X \in g^{p,-p}$, $XH^{r,s} \subset H^{r+p, s-p}$. 

The Lie algebra $b$ of $B$ consists of all those $X \in g$ that preserves the reference Hodge
filtration. i.e
\[
    b = \bigoplus_{p\ge 0} g^{p,-p}
\]
Let $v$ be the Lie algebra of $V = G_{\real}\cap B$, then
\[
    v = g_0 \cap b = g_0\cap b \cap \bar{b} = b\cap g^{0,0}
\]

The holomorphic tangent space of $\check{D} \cong G_{\complex}/B$ at 
the base point is naturally isomorphic to $g/b$. Under this iso, the
action of the isotropy group $B$ on the tangent space correspond to
the adjoint action of $B$ on $g/b$. Consequently, the holomorphic tangent
bundle $T \rightarrow \check{D}$ coincides with the vector bundle associated
to the holomorphic principal bundle
\[
    B \rightarrow G_{\complex} \rightarrow G_{\complex}/B \cong \check{D}
\]
\[
    b \oplus g^{-1,1}/b 
\]
Defines a $B-$invariant subspace of $g/b$. Therefore, it gives rise to a $B$
invariant subbundle $T_h(\check{D})$. Call it horizontal tangent bundle.

A holomorphic map $\Psi: M \rightarrow \check{D}$ is said to be horizontal if 
at each point $M$, the induced map on tangent space takes value in $T_h(\check{D})$.

Construct $T_h(D)$ by restriction. $G_{\complex}$-invariance of $T_h(\check{D})$ 
implies $G_{\real}$-invariance of $T_h(D)$. 

\subsection{Alternate description of the horizontal tagent bundle}
\begin{lemma}
At each point $c \in \check{D}$, correpsonding to the filtration $F(c)$, a vector
field $X \in g$ takes value in the horizontal tangent space if and only if
$X$ regarded as an endomorphism of $H_{\complex}$, maps $F^p$ to $F^{p-1}$
\end{lemma}
\begin{proof}
At the base point, the value of $X$ lie in
\[
    b\oplus g^{-1,1}/b
\]
Therefore, the value of $X$ can be represented by somebody in $g^{-1,1}$
\end{proof}

By construction, $\check{D}$ carries a tautological complex vector bundle $H_{\complex}(\check{D})$,
with fibre $H_{\complex}$. The global filtration $\mathbb{F}$ of the bundle, restrict to the 
fitration of $H_{\complex}$ corresponding to the base point. Let
\[
    \nabla: \mathbb{H_{\complex}} \rightarrow \mathbb{H_{\complex}}\otimes \Omega_{\check{D}}
\]
be the flat connection. 
\newcommand{\hd}{H_{\complex{\check{D}}}}
Let $p:\hd \rightarrow \hd/F^p$, the composition
\[
    p\circ\nabla F^p \rightarrow \hd/F^p \otimes \Omega_{\check{D}}
\]
is $O_{\check{D}}$-linear. 

Consider the holomorphic map $\Psi: M \rightarrow \check{D}$ of a complex manifold $M$ into 
$\check{D}$. The vector bundle $H_{\complex}(\check{D})$ pulls back to the trivial bundle
$\mathbb{H_{\complex}} \rightarrow M$

A holomorphic map $\Psi: M \rightarrow D$ is horizontal if and only if 
\[
    \nabla(F^p) \subset F^{p-1}\otimes\Omega_{D}
\]

\newcommand{\VHS}{\mathbb{H}_{\complex}}
Now, consider a variation of Hodge structure $\{M, \VHS, F\}$. Let $\pi: \tilde{M} \rightarrow M$
denote the universal covering of $M$. The vector bundle $\VHS$ pullback to a trivial bundle
$\tilde{M}\times H_{\complex}$. The filtration also pulls back. Therefore, for each point 
$m \in \tilde{M}$, we get a Hodge structure on $H_{\complex}$ of weight rank $\VHS$. So we get
a map
\[
    \tilde{\Phi}: \tilde{M} \rightarrow D
\]
Variation of Hodge structure has the attribute of transversality. Therefore, the map $\tilde{\Phi}$
is automatically horizontal.

$\VHS.H_{\integer}$ and $\VHS.S$ are flat, therefore the action of $\pi_1(M)$ preserves both. 

One can think of $\VHS$ on $M$ as a quotient from the universal cover in the first place.

The subgroup
\[
    \Gamma = \phi(\pi_1(M)) \subset G_{\integer}
\]
is called the monodromy group of the variation of Hodge structure. 

By construction of $\tilde{\Phi}$, if two points of $\tilde{M}$ are 
related by some $\sigma \in \pi_1(M)$, the corresponding Hodge structures are related by
$\psi(\sigma)$.
\[
    \tilde{\sigma a} = \psi(\sigma)\circ\tilde{\Psi}(a), a \in \tilde{M}, \sigma \in \pi_1(M)
\]
Therefore, $\tilde{\Phi}$ descends to a mapping
\[
    \Phi: M \rightarrow D/\Gamma
\]
This map is called Griffith's period mapping for variaiton of Hodge structure. 

\begin{theorem}(Griffith)
The period mapping is holomorphic, locally liftable to $D$, and local liftings are horizontal
\end{theorem}

\section{Nilpotent Orbit Theorem}
Let $\phi: \Delta^* \rightarrow D/\Gamma$ be the period mapping of VHS on $\Delta^*$
Let $\pi: U \rightarrow \Delta^*$ be the universal covering map. $\phi\circ\pi$ is locally
liftable, and since $U$ is simply connected, it is globally liftable. We have the following diagram
\[
\begin{tikzcd}
    & U \arrow{r}{\tilde{\Phi}}\arrow{d}{\pi} 
        & D \arrow{d} \\
    & \Delta^* \arrow{r}{\Phi} 
        & D/\Gamma \\
\end{tikzcd}
\]
One can choose an element $\gamma \in \Gamma\subset G_{\integer}$, such that
\[
    \tilde{\Phi}(z+1) = \gamma\circ\tilde{\Phi}(z)
\]
\begin{lemma}(Borel)
All eigenvalues of $\gamma$ are roots of unity
\end{lemma}
\begin{proof}
Computed with respect to Poincar\'e metric 
\[
    \frac{dx^2 + dy^2}{y^2}
\]
on $U$. the points $in$ and $in + 1$ have the distance $1/n$.
According to 3.17, if a $G_{\real}$-invariant Riemannian distance function $d$
on $D$ is suitably renormalized, $\tilde{\Phi}$ will not increase distances.
When the identification $D \cong G_{\real}/\Gamma$ is made as in 3.6, each
pf the image points $\tilde{\Phi}(in)$ has a coset rep $g_nV$. So
$\tilde{\Phi}(in + 1) = \gamma\tilde{\Phi}(in)$ corresponds to the coset $\gamma g_nV$.
Thus
\begin{align*}
    d(\inverse{g_n}\gamma\g_n V, eV) & = & d(\gamma g_nV, g_nV) \\
                                     & = & d(\tilde{\Phi}(in+1), \tilde{\Phi}(in))
                                     & \le & \frac{1}{n}
\end{align*}
This means the congugacy class of $\gamma$ converge to a point in the compact subgroup
$V \subset G_{\real}$. To show eigenvalues of $\gamma$ have abosolute value 1, 
it is enough to show that the eigenvalues for each each element in $V$ have absolute 
value 1. As $\gamma$ is integer-valued matrix with respect to the basis $H_{\integer}$
it follows that the eigenvalues of $\gamma$ must be root of unity. 
\end{proof}


Write $\gamma = \gamma_u\gamma_s$, where $\gamma_u$ is unipotent, and $\gamma_s$
is semi-simple. Let $m$ be the least integer such that $\gamma_s^m = 1$. Define
\begin{align*}
    N & = \log\gamma_u = \sum\limits_{k\ge 1}(-1)^{k+1}\frac{1}{k}(\gamma_u-1)^k \\
      & = \frac{1}{m}\log\gamma = \sum\limits_{k\ge 1}(-1)^{k+1}\frac{1}{k}(\gamma - 1)^k 
\end{align*}

Given $\tilde{\phi}: U \rightarrow D$, I want to define a map invariant under
$z \mapsto z + 1$, so that it drop to a map $\Delta^* \rightarrow D$. 
\[
    \tilde{z} = \exp{-mzN}\circ\tilde{\phi}(mz)
\]
has this property. Denote the descent of this map by $\Phi$.

\begin{theorem}(One-Variable case of Nilpotent Oribit Theorem)
The mapping $\Phi$ can be continued holomorphically over the puncture of $\Delta^*$.
The point $a = \Phi(0) \in \check{D}$ is a fixed point of $\gamma_s$. For a suitable
constant $\alpha \ge 0$, $Im z \gr \alpha$ implies $\exp(zN)\circ a \in D$.
Perhaps after increasing the constant $\alpha$, and for a suitable choice of $\beta \ge 0$
$Im z >\alpha$ also implies the inequality
\[
    d(\exp(zN)\circ a, \tilde{\phi}(z)) \le (Im z)^{\beta}\exp(-2\pi\inverse{m}Im z)
\]
here $d$ denote a $G_{\real}$-invariant Riemannian distance function used to prove 
the lemma of Borel. The mapping $z \mapsto \exp(zN)\circ a$ is horizontal.
\end{theorem}



\end{document}
