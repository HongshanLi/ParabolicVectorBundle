\documentclass{article}
\usepackage{mathpkgs}
\begin{document}
\begin{definition}
A normal variety $X$ of dimension $n$ has only terminal singularities if
\begin{enumerate}
\item The canonical divisor is $\rational$-Cartier
\item There exists a projective birational morphism $f: V \rightarrow X$ from
    a nonsingular variety $V$ such that in the ramification formula
\[
    K_V = \pullback{f}K_X + \sum a_iE_i
\]
all coefficients for the exceptional divisors are strictly positive.
\end{enumerate}
\end{definition}

\begin{definition}
A log pair $(X, D)$ is a pair consisting of a normal variety $X$ and a boundary
$\rational$-divisor $D = \sum d_iD_i$ such that $0 \le d_i \le 1$. We call
$K_X + D$ the log canonical divisor of the pair $(X, D)$.
\end{definition}


\begin{definition}
A logarithmic pair $(X, D)$ has only log terminal(respectively log canonical) 
singularities, if 
\begin{enumerate}
\item the log canonical divisor $K_X + D$ is $\rational$-Cartier
\item there exists a projective birational morphism
\[
    f: V \rightarrow X
\]
from a nonsingular variety $V$ such that
\[
    D_V = \sum E_i + \inverse{f_*}D
\]
is a divisor of normal crossings, where $E_i$ are the exceptional divisors
for $f$ and $\inverse{f_*}$ is the strict transform of $D$, and such that
in the logarithmic ramificaition formula
\[
    K_V + D_V = \pullback{f}(K_X + D) + \sum b_iE_i
\]
all $b_i >0$(respectively $b_i \ge 0$)
\end{enumerate}
\end{definition}

\begin{definition}
The log pair $(X, D)$ has Kawamata log terminal singularity if for the boundary 
divisor $D = \sum d_kD_k$, we have 
\[
    0 \le d_k < 1
\]
\end{definition}


