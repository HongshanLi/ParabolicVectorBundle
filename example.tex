\documentclass{article}
\usepackage{mathpkgs}
\usepackage{mathtools}
\DeclarePairedDelimiter\ceil{\ceil}{\rceil}
\DeclarePairedDelimiter\floor{\lfloor}{\rfloor}

\renewcommand{\vector}[2]{
\left[\begin{array}{c}
    {#1} \\
    {#2}
    \end{array}\right]
}
\renewcommand{\matrix}[4]{
\left[\begin{array}{cc}
    {#1} & {#2} \\
    {#3} & {#4}
    \end{array}\right]
}
\begin{document}

Let $X$ be a complex manifold, 
and let $D \subset X$ be a normal crossing divisor.
\newcommand{\V}{\mathcal{V}}
Let $\V$ be a quasi-unipotent local sytem on $X - D$. 
Then, the canonical extension $V_X$ of $\V$ has nontrivial parabolic structure.
Let $p: Y \rightarrow X$ be a branched cyclic cover such that $\inverse{p}\V$ is
unipotent. Then, we will show that the orbifold bundle on $Y$ corresponding to $V_X$
is the canonical extension of $\inverse{p}\V$. 

I will use the following example to illutrate the ideas. 
Let $X = \Delta$, $D$ be the origin, and $\V = (\mathbb{Z}^2, T)$,
where $T$ is the diagonal quasi-unipotent matrix
\[
    T = \left[\begin{array}{cc}
        \epsilon_3 & 0 \\
        0 & \epsilon_5 \\
        \end{array}\right]
\]
where $\epsilon_i$ is a primitive $i$-th root of unity. 

\section{Construction of the canonical extension of $\V$}
In this note, by canonical extension, we mean the extension $(V, \nabla)$ of $\V$
such that the eigenvalues of $\text{Res}(\nabla)$ lie in $[0, 1)$

Let 
\[
    M = -\log T = - \left[\begin{array}{cc}
                       \log \epsilon_1 & 0\\
                        0 & \log \epsilon_5
                        \end{array}\right]
\]
And we use the branch one step before the principal branch for both log functions, \emph{i.e.}
\[
    M = -\left[\begin{array}{cc}
            \frac{2\pi i}{3} - 2\pi i & 0 \\
            0 & \frac{2\pi i}{5} - 2\pi i
            \end{array}\right]
\]
Let $\mathbb{H} \rightarrow \Delta^*$ be the universal covering map. Let $z$ and $t$ be 
the coordinate on $\Delta$ and $t$, we have $z = \exp{2\pi it}$. 
Let $<e_1, e_2>$ be the standard basis for $\mathbb{Z}^2$. Consider the vector valued 
functions on $\mathbb{H}$
\[
    s_1(t) = \exp(Mt)\otimes e_1, s_2(t) = \exp(Mt)\otimes e_2
\]
\emph{i.e.}
\[
    s_1(t) = \vector{\exp{\frac{4\pi it}{3}}}{0}
    s_2(t) = \vector{0}{\exp{\frac{8\pi it}{5}}}
\]

Use $z = \exp{2\pi it}$, we can regard $s_1$ and $s_2$ as multi-valued functions on $\Delta^*$
\[
    s_1(x) = \vector{x^{\frac{2}{3}}}{0},
    s_2(x) = \vector{0}{x^{\frac{4}{5}}}
\]


Let $V = O_X\cdot s_1(x)\oplus O_X\cdot s_2(x)$.
Define $\nabla : V \rightarrow V\otimes\Omega_X(\log D)$, such that the connection matrix
is given by
\[
    \frac{M}{2\pi i}\cdot\frac{dx}{x}
\]
Then,
\[
    \text{Res}(\nabla) = \left[\begin{array}{cc}
                            \frac{2}{3} & 0 \\
                            0 & \frac{4}{5}
                            \end{array}\right]
\]
This verifies that the $(V, \nabla)$ is the canonical extension of $\V$.
Use $<e_1, e_2>$ as the standard basis of $\mathbb{Z}^2$, the inclusion map
$\V \rightarrow V|_{\Delta^*}$ is given by
\[
    e_1 \mapsto x^{-\frac{1}{3}}s_1, e_2 \mapsto x^{-\frac{4}{5}}s_2
\]
And we can check that the images of $e_1$ and $e_2$ are indeed flat sections of $\nabla$.


\subsection{The action of $x$ on $V_X$}
This section will explain how $x$ acts on $V_X$ and send $V_X$ to an extension of $\V$ whose
residue has eigenvalues in $[1, 2)$. We will this action latter to explain why the morphism
I talked about in our last meeting is a morphism of \emph{parabolic} bundles.

$x$ has a natural action on $s_1$ and $s_2$
\[
    x\cdot s_1 = \vector{x^{\frac{4}{3}}}{0},
    x\cdot s_2 = \vector{0}{x^{\frac{9}{5}}}
\]

We would have ended up those sections if we used $-4\pi i$ branch for our $M$. More precisely,
if we set 
\[
    M = - \matrix{\frac{2\pi i}{3} - 4\pi i}{0}{0}{\frac{2\pi i}{5} - 4\pi i}
\]
at the very begining. Then, we will get an extention $(V^1, \nabla^1)$ with the generating 
sections
\[
    x\cdot s_1 \text{and} x\cdot s_1
\]
The connection $\nabla^1$ is defined by
\[
    \frac{M}{2\pi i}\cdot\frac{dx}{x}
\]
So
\[
    \text{Res}(\nabla) = \matrix{\frac{4}{3}}{0}{0}{\frac{9}{5}}
\]

\subsection{The parabolic structure on $V_X$}
$\text{Res}(\nabla)$ defines an endomorphism of $V_X|_D$, the parabolic structure is given
by the generalized eigenspaces of this endomorphism. In our example, the story is quite 
simple. The eigenvalues of $\text{Res}(\nabla)$ are $\frac{2}{3}$ and $\frac{4}{5}$, 
the corresponding eigenspaces are $\complex\cdot s_1$ and $\complex\cdot s_2$. 
The filtration on $V_X$ is
\[
    V_X = F_1 \supset F_2 = O_X\cdot s_2 \supset V_X(-D)
\]
and the weights are $\frac{2}{3}, \frac{4}{5}$. 


\section{On the branch Cover}
Let $p : Y \rightarrow X$ be the branched coverd defined by $y^{15} = x$. 
Let $\Gamma$ be the Galois group of this covering map.
Then, $\inverse{p}\V = (\mathbb{Z}^2, T^{15})$ is the trivial local system on $Y$.
So the canonical extension $V_Y$ of $\inverse{p}\V$ is the trivial bundle with the trivial 
connection, \emph{i.e.}
\[
    V_Y = O_Y\cdot e_1 \oplus O_Y\cdot e_2
\]
where $<e_1, e_2>$ is the standard basis of $\integer^2$. The connection $\nabla_Y$ on $V_Y$
has the trivial connection matrix.

$V_Y$ inherits an $\Gamma$-action from $O_Y$, and clearly this makes $V_Y$ into an equivariant
bundle. 

\textbf{The $\Gamma$-invariant part of $\pushforward{p}V_Y$}\newline
$e_1$ and $e_2$ are constant sections on $Y$, so $\Gamma$ has trivial action on them.
So to compute $(\pushforward{p}V_Y)^{\Gamma}$, we only need to compute 
$(\pushforward{p}O_Y)^{\Gamma}$. As an $O_X$-module, $\pushforward{p}O_Y$ looks like
\[
    O_X\cdot 1 \oplus O_X\cdot y \oplus O_X\cdot y^2 \cdots \oplus O_X\cdot y^{14}
\]
where $y$ is a local coordinate on $Y$. Each $y^i$ has a nontrivial $\Gamma$-action,
\emph{i.e.} if $g$ is a generator of $\Gamma$, then $g\cdot y^i = (g\cdot y)^i$.
So the only $\Gamma$-invariant part of $\pushforward{p}O_Y$ is $O_X\cdot 1$. Hence,
\[
    (\pushforward{p}V_Y)^{\Gamma} = O_X\cdot e_1 \oplus O_X\cdot e_2
\]
\newcommand{\invariant}{(\pushforward{p}V_Y)^{\Gamma}}
\textbf{The morphism $\alpha: \invariant \rightarrow V_X$}\newline
It is natural to define $\alpha$ as
\[
    e_1 \mapsto s_1, e_2 \mapsto s_2
\]
\newcommand{\tD}{\tilde{D}} 
Let $\tD = \inverse{p}D$. The parabolic structure on $\invariant$ is given by
\[
    \invariant_t = (\pushforward{p} V_Y\otimes O_Y(\floor*{t\tD}))^{\Gamma}
\]

    

  


\end{document}
                
