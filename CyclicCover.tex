\documentclass{article}
\usepackage{mathpkgs}
\begin{document}
Let $D$ be a smooth divisor of a smooth projective variety $X$. 
If I want a cyclic cover branched over $D$, then I can just through in
$n$-th root of local equations of $D$ in the function field $\complex(X)$ then
normalize $X$ in $\complex(X)(\text{n-th root of local equation of $D$})$.
\[
    \pi: \tilde{X} \rightarrow X
\]
And hope that $D$ is the only place that $\pi$ branches. But local equation
of $D$ is just a rational function on $X$. Somewhere on $X$ it will pick up
a pole, and naturally $\pi$ branches over the pole just as branches over the zeros
of the rational function. That is why during the construction of cyclic cover, one
picks up some other branch divisors.

This is why in Kawamata's construction of cyclic cover, we cannot just throw in
$f_1^{1/n_1}$, where $f_1$ is the local equation of $D_1$, because $f_1$ might 
have a pole on $D_2$. We cannot just throw in $f_1^{1/n_1}$, because we will 
pick up the branching over the pole of $f_1^{1/n_1}$, if we are unlucky, the 
poles might not be normal crossing. That is why we need to control what pole
should be in the first place, by choosing all these $H_i$'s.
 

Let $D = D_1 + D_2 +\cdots + D_n$ be a normal crossing divisor. If we want a 
cyclic cover 
\[
    \pi: \tilde{X} \rightarrow X
\]
that branches over $D_i$ of degree $n_i$. Then, the naive idea would be 
to throw in $f_i^{1/n_i}$, where $f_i$ is the local equation of $D_i$. 
Obviously, this is a stupid idea, because $\pi$ will be generically etale
and it cannot be generically etale of degree $n_1$ and $n_2$ at the same
time. 

Consider the following map $\phi$
\begin{align*}
    \Delta\times\Delta & \rightarrow & \Delta\times\Delta \\
      y_1, y_2         & \mapsto     & y_1^n, y_2
\end{align*}
Let $D_1$ and $D_2$ be divisors downstairs defined by $y_1=0$ and $y_2=0$.
Then, $\pullback{\phi}D_1 = n_1D_1$, and $\pullback{\phi}D_2 = D_2$. To make
a cyclic cover branched over $D_2$, we fix the first coordinate, and raise the 
power of the second coordinate. Call the following map $\phi$
\begin{align*}
    \Delta\times\Delta & \rightarrow & \Delta\times\Delta \\
      y_1, y_2         & \mapsto     & y_1^{n_1}, y_2^{n_2}
\end{align*}
Then
\[
    \pullback{\phi}D_1 = n_1D_1, \pullback{\phi}D_2 = n_2D_2
\]


In the proof Theorem 17 of Ka, we could
\end{document}
 


